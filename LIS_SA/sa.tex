\documentclass{article}
\usepackage[utf8]{inputenc}
\usepackage{amsmath}
\usepackage{graphicx}
\usepackage[procnames]{listings}
\usepackage{color}
\usepackage{indentfirst}
\usepackage{xcolor}
\usepackage{sectsty}
\usepackage[explicit]{titlesec}


\usepackage[normalem]{ulem}
\title{SA/SD}
\begin{document}
\maketitle
\part{System Analysis}
\section{Feasability Study}
\subsection{Understanding the problem}
In this section, we make an effort to understand the purpose of the software.
\\
The LIS(Library Information Software) is intended to ensure a systematic manner of maintaining the database of a library and easing the job of a librarian.The LIS will help maintain a record of books in the library and members registered with the library.The software aims at hassle-free handling of basic library functions like issue/reserve/return books as well as giving the freedom to the librarian to add and delete books and members to the database.
	Extending further the software can be used by the librarian to decide on disposing of books which have not been issued for a long time. This software also helps to keep track of statistics of issue of a particular book ver a period of time.Since everything is done
on the computer, it is easy to record all data and there are minimal chances of
inconsistencies or ambiguities.
\subsection{Scope of the Problem}
In this section,we list the various functions performed by the software:
\begin{itemize}
\item Add books to the database
\item Modify existing records of books
\item Delete unused books from database
\item Add users to the database
\item Modify existing details of users
\item Remove user from the database
\item Query regarding books on ISBN number or name
\item Query regarding user on user id or name
\item Record issue,return and reservation of books
\item Calculation of penalty on overdue books
\item Maintains statistics of all books 
\end{itemize}
\subsection{Analysing stakeholders}
The various stakeholders are:
\begin{itemize}
\item Librarian
\item Member
\end{itemize}
These stakeholders have been clearly described with their attributes and methods in the SRS.
\subsection{Defining Alternatives}
\subsubsection{Connection between software and librarian}
\subsubsection{Connection between software and member}
\subsubsection{Hardware Infrastructure}
\subsubsection{Software Infrastructure}
\subsubsection{Technology Used}
\subsubsection{Security}
\subsection{Defining Criteria to evaluate}
\subsection{Report}
\section{UML Diagrams}

\subsection{Refinement of Use Case Diagram}
\subsection{Refinement of Class Diagrams}
\subsection{Sequence Diagram}
\subsection{Collaboration Diagram}
\subsection{Statechart Diagram}
\subsection{Activity Diagram}
\section{System Parameters}
\subsection{Platform}
\subsection{Language}
\subsection{Build System}
\subsection{Libraries}
\subsection{Sizing}
\subsection{Performance}
\section{Limitations and Exceptions}
\section{Other Information about analysis}
\part{System Design}
\section{Refined System Parameters}
\subsection{Global System Architecture}
The overall system architecture is a 2-tier architecture which includes client at one end and the
database at the other. There is no server based middle tier in the software being designed.
\subsection{Platform}
\subsubsection{Hardware}
\subsubsection{Software}
\subsubsection{Networking}
\subsection{Software Architecture}
Object-oriented architecture forms the basis of the LIS. In this style data representations and
their associated primitive operations are encapsulated in an abstract data type or object. The
components of this style are the objects—or instances of the abstract data types. Objects
interact through function and procedure invocations.
Two important aspects of this style are
A. that an object is responsible for preserving the integrity of its representation (usually by
maintaining some invariant over it), and
B. that the representation is hidden from other objects.
Thus the aspects of OOA mentioned justify our choice.
\subsubsection{Details}
\subsubsection{Justification}
\section{Database Design}
\section{Design Details}
\subsection{Refinement of UML diagrams}
\subsection{Prototype Design}
\subsection{Design I/O procedures and user interfaces}
\subsection{Design of classes in target language}
\subsection{Exception Design}
\section{Adoptable Practices}
\section{Any Other Information Of previous Stages}
\end{document}
